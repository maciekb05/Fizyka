\documentclass [a4paper,11pt]{article}
\usepackage{amssymb}
\usepackage{amsthm}
\usepackage[intlimits]{amsmath}
\usepackage[polish]{babel}
\usepackage[utf8]{inputenc}
\usepackage[T1]{fontenc}
\frenchspacing
\usepackage{indentfirst}
\usepackage{graphicx}
\usepackage{subfig}
\usepackage{mathptmx}
\usepackage{geometry}
\usepackage{wrapfig}
\usepackage{enumitem}
\usepackage{tabularx}

\title{Moduł Younga}
\author{Pęcak Tomasz, Bielech Maciej}

\begin{document}
	
	\renewcommand*{\figurename}{Rysunek} 
	\newgeometry{tmargin=2cm, bmargin=2cm, lmargin=2cm, rmargin=2cm}
	
	\linespread{1.5}
	\selectfont

	\begin{table}[]
		\centering
		\begin{tabular}{lllllll}
			\cline{1-6}
			\multicolumn{1}{|c|}{\begin{tabular}[c]{@{}c@{}}EAiIB\\ Informatyka\end{tabular}} & \multicolumn{2}{l|}{\begin{tabular}[c]{@{}l@{}}Pęcak Tomasz\\ Bielech Maciej\end{tabular}} & \multicolumn{1}{c|}{\begin{tabular}[c]{@{}c@{}}Rok\\ II\end{tabular}} & \multicolumn{1}{c|}{\begin{tabular}[c]{@{}c@{}}Grupa\\ 3a\end{tabular}} & \multicolumn{1}{c|}{\begin{tabular}[c]{@{}c@{}}Zespół\\ II\end{tabular}} &  \\ \cline{1-6}
			\multicolumn{1}{|c|}{\begin{tabular}[c]{@{}c@{}}Pracownia\\ FIZYCZNA\\ WFiIS AGH\end{tabular}} & \multicolumn{4}{l|}{\begin{tabular}[c]{@{}l@{}}Temat:\\ \textbf{Mostek Wheatstone'a} \end{tabular}} & 
			\multicolumn{1}{l|}{\begin{tabular}[c]{@{}l@{}}nr ćwiczenia:\\ 32\end{tabular}} &  \\ \cline{1-6}
			\multicolumn{1}{|l|}{\begin{tabular}[c]{@{}c@{}}Data wykonania:\\ 11.11.2017\end{tabular}} & \multicolumn{1}{c|}{\begin{tabular}[c]{@{}c@{}}Data oddania:\\ 14.11.2017\end{tabular}} & \multicolumn{1}{l|}{\begin{tabular}[c]{@{}l@{}}Zwrot do poprawki:\\ \phantom{data poprawki}\end{tabular}} & \multicolumn{1}{l|}{\begin{tabular}[c]{@{}l@{}}Data oddania:\\  \phantom{data oddania}\end{tabular}} & \multicolumn{1}{l|}{\begin{tabular}[c]{@{}l@{}}Data zaliczenia:\\  \phantom{data zaliczenia}\end{tabular}} & \multicolumn{1}{l|}{\begin{tabular}[c]{@{}l@{}}OCENA:\\ \phantom{ocena}\end{tabular}} &  \\ \cline{1-6} 
		\end{tabular}
	\end{table}
	 \hspace{5mm}

	\section{Wstęp}
	\section{Wykonanie ćwiczenia}
	Ćwiczenie wykonywaliśmy dla pięciu rezystorów, połączenia szeregowego pierwszego i drugiego rezystora, połączenia równoległego pierwszego i drugiego rezystora, połączenia mieszanego (w tym połączniu rezystor trzeci został połączony szeregowo z równoległym połączeniem pierwszego i drugiego rezystora).
	Dla każdego układu wykonano następujące kroki:
	\begin{itemize}
		\item W pierwszym kroku ustawiono kontakt ślizgowy listwy z drutem oporowym na środek (tak, aby $a=b$).
		
		\item Następnie, dostosowywano rezystancję opornicy dekadowej, tak aby wskazówka mikroamperomierza była wyzerowana.
		
		\item Kolejnym krokiem było zmienianie rezystancji opornicy dekadowej i przestawianie kontaktu ślizgowego, tak aby wskazówka mikroamperomierza wskazywała 0.Wykonano 10 takich zmian zapisując położenie kontaktu ślizgowego ($a$).
		
		
		\item Wyniki zapisano w tabelkach.
	\end{itemize}

	
	\section{Opracowanie danych pomiarowych}\label{sec:opr}
	\subsection{Pomiary i ich niepewności.}
		

 
	
	\subsection{Opracowanie danych.}\label{sec:drm}
	\begin{enumerate}[label=\alph*)]
		
		\item Analiza błędów.
		
		Stwierdzono wystąpienie błędów grubych, które wyraźnie odstają od średniej. Zaznaczono je w tabelkach kolorem czerwonym.
		
		\item Obliczenie wartości rezystancji połączeń szeregowego, równoległego i mieszanego korzystając z wyników pomiarów $R_1$, $R_2$ i $R_3$ i ich niepewności.
		

		
		Niepewności wyliczenia rezystancji zastępczych obliczone zostały z wykorzystaniem prawa przenoszenia niepewności za pomocą następujących wzorów:

		

	
	\end{enumerate}
	
	\section{Podsumowanie}
	\begin{center}
		\begin{tabular}{|c|c|c|c|c|}
			\hline Opis wielkości & Opór wyznaczony $R_x$ $[\mathrm{\Omega}]$ & $u(R_x)$ $[\mathrm{\Omega}]$ &  $ \frac{u(R_x)}{R_x} $ $[\%]$ \\
			\hline $R_1$ & 12,553 & 0,024 & 0,19  \\
			\hline $R_2$ & 34,36 & 0,15 & 0,45  \\
			\hline $R_3$ & 70,31 & 0,15 & 0,21  \\
			\hline $R_4$ & 38,04 & 0,13 & 0,35  \\
			\hline $R_5$ & 104,48 & 0,18 & 0,17  \\
			\hline $R_{z_1}$ & 46,82 & 0,17 & 0,37  \\
			\hline $R_{z_2}$ & 9,567 & 0,031 & 0,33  \\  
			\hline $R_{z_3}$ & 79,93 & 0,31 & 0,39  \\ 
			\hline 
		\end{tabular} 
	\end{center}
\vspace{1em}



\end{document}